% tuna pasta.tex
% recipe file for my tuna pasta

\begin{recipe}
[%
% ~~ RECIPE OVERVIEW: ~~
	preparationtime = {\unit[5 - 10]{min}},
	bakingtime = {\unit[20 - 30]{min}},
	portion = {\unit[1]{big pot}},
	source = {Ana},
	vegetarian = {no},
	vegan = {no}
]
%
% ~~ RECIPE NAME: ~~
{Tuna Pasta}


% ~~ PICTURES ~~
% Appear on top of the page (optional)
% Insert path\to\filename
\graph
{ 
    small=tuna,     % small picture
    big=pasta  % big picture
}


% ~~ INTRODUCTION ~~
% History/anecdote on the recipe (optional)
\introduction
{ 
My ultimate comfort food: a nutmegy and saucy tuna pasta.
}


% ~~ INGREDIENTS ~~
% List as '& quantity & item &\\'
% Caution: no return ('\\') after last item!
% [v1.1] Multi-columns: '& quantity1 & item1 && quantity2 & item2 ... &\\'
\ingredients[2] % [v1.1] number of ingredient columns 
{%
    1 & onion\\
    1 & tomatoes (or half a can of tomatoes)\\
    3 & garlic cloves\\
    1/2 & red pepper\\
    \unit[50]{mL} & milk (or any substitute)\\
    \unit[1/2]{cup pp} & any short-length pasta you like\\
    \unit[2]{cans} & tuna chuncks (pref. in water)\\
    some & fresh coriander (optional)\\
    some & fresh basil\\
    about \unit[1/4]{tsp} & nutmeg\\
    pinch & cumin\\
    about \unit[1/4]{tsp} & Piri piri\\
    or 1/2 & chili pepper\\
    some & olive oil\\
    \unit[1/4]{tsp} & salt\\
}


% ~~ PREPARATION ~~
% Steps as '\step{whatever the step is}'
% [until v1.1] Caution: for optimal floating, steps need to be at >= 2
% 				lines long! -> if one-liner, add a return ('\\') 
\preparation{%
    \step{Dice the onions and the garlic. Peel the tomatoes and chop them in small cubes. Chop the peppers into small pieces and dice the herbs and the chilli pepper in case you're using it.}
    \step{Fry the onions and the garlic in some olive oil.}
    \step{Once they have started to golden, add the red peppers.}
    \step{Add the tomatoes, the herbs, the spices and one of the tuna cans and let it cook for a few minutes, stirring constantly to keep it from sticking.}
    \step{Add 2 cups of boiling water and the milk. Let it cook for about \unit[5 - 10]{minutes}.}
    \step{Add enough water to cook the pasta you want to make. Taste and adjust salt and other spices to your taste.}
    \step{Let it cook, adding more water when necessary, until the flavour is to your taste (about \unit[5 - 10]{minutes} should be enough.)}
    \step{Add the pasta, and cook according to the package, stirring constantly. Add more water if necessary or if you want to have more sauce.}
    \step{Once the rice is cooked, turn off the heat, serve warm and enjoy!}
} 


% ~~ SUGGESTIONS ~~
% Chef suggestions (optional)
\suggestion
[Pepper sauce] % suggestion title
{ 
To make it even tastier, use \unit[1 - 2]{tsp} of the red pepper sauce instead of the half pepper!
}


% ~~ HINT ~~
% Specific hint (optional)
\hint
{
There are a few tricks to easily peel the tomatoes. I usually keep some frozen and them just take them out, put them in a bowl of warm water for a few minutes and the skin comes off quite easily.
However, if you don't want to keep frozen tomatoes a good trick is to make an X on the bottom and put them in a pot of boiling water for no more than a minute. Then put them into a bowl of cold water (or an ice bath), taking them immediately back out. Peeling them should be easy now!
}

\end{recipe}

% end of recipe file