% vegan pataniscas.tex
% recipe file for a vegan version of the Portuguese pataniscas

\begin{recipe}
[%
% ~~ RECIPE OVERVIEW: ~~
	preparationtime = {\unit[1]{h} \unit[30]{min}},
	bakingtime = {\unit[30]{min}},
	portion = {\portion{8 - 12}},
	source = {Ana},
	country = {Portugal},
	vegetarian = {yes},
	vegan = {yes}
]
%
% ~~ RECIPE NAME: ~~
{Pataniscas}


% ~~ PICTURES ~~
% Appear on top of the page (optional)
% Insert path\to\filename
\graph
{ 
    %small=pic/glass,     % small picture
    big=vegan_pataniscas  % big picture
}


% ~~ INTRODUCTION ~~
% History/anecdote on the recipe (optional)
\introduction
{ 
A vegan version of the classic Portuguese pataniscas made up by moi. They go especially well with Arroz de feijão (rice and beans) or some other form of rice.
}


% ~~ INGREDIENTS ~~
% List as '& quantity & item &\\'
% Caution: no return ('\\') after last item!
% [v1.1] Multi-columns: '& quantity1 & item1 && quantity2 & item2 ... &\\'
\ingredients[2] % [v1.1] number of ingredient columns 
{%
    1 & courgette\\
    1 & aubergine\\
    2 & minced garlic cloves\\
    \unit[1 + 1/2]{cups} & all-purpose flour\\
    \unit[1]{cup} & corn starch\\
    \unit[2]{cups} & almond milk (or another milk-substitute)\\
    \unit[5]{tbsp} & linseed (in \unit[15]{tbsp} of warm water)\\
    \unit[1/4]{cup} & diced fresh flat-leaf parsley\\
    a lot of & salt\\
    enough & vegetable oil (to deep fry)\\
}


% ~~ PREPARATION ~~
% Steps as '\step{whatever the step is}'
% [until v1.1] Caution: for optimal floating, steps need to be at >= 2
% 				lines long! -> if one-liner, add a return ('\\') 
\preparation{%
    \step{Peel and grate the courgette and aubergine, cover in a lot of salt and leave for about \unit[1]{h}. Rinse to get rid of the extra salt.}
    \step{Mix the flour, linseed and the milk.}
    \step{Add the vegetables, parsley and garlic and mix again.}
    \step{Fill a frying pan with oil and heat it up until the oil is hot.}
    \step{Take out bits of the batter with a ladle (about 2/3 full) and deep fry them until golden.}
    \step{Set them aside in some paper towels to soak up the extra grease.}
    \step{Serve with whatever side you have and enjoy!}
} 


% ~~ SUGGESTIONS ~~
% Chef suggestions (optional)
\suggestion
[Chickpea flour] % suggestion title
{ 
Instead of corn starch, you can also use chickpea flour. This will also make them more yellow, resembling the colour of the normal ones.
}


\end{recipe}

% end of recipe file