% breadsticks.tex
% recipe file for homemade breadsticks

\begin{recipe}
[%
% ~~ RECIPE OVERVIEW: ~~
	preparationtime = {\unit[1]{h}\unit[30]{min}},
	bakingtime = {\unit[20]{min}},
	%portion = {\portion{5 - 8}},
	source = {Ana},
	vegetarian = {yes},
	vegan = {yes}
]
%
% ~~ RECIPE NAME: ~~
{Breadsticks}


% ~~ PICTURES ~~
% Appear on top of the page (optional)
% Insert path\to\filename
\graph
{ 
    %small=pic/glass,     % small picture
    big=breadsticks  % big picture
}


% ~~ INTRODUCTION ~~
% History/anecdote on the recipe (optional)
\introduction
{ 
These breatsticks are soft on the inside but crunchy on the outside.
}


% ~~ INGREDIENTS ~~
% List as '& quantity & item &\\'
% Caution: no return ('\\') after last item!
% [v1.1] Multi-columns: '& quantity1 & item1 && quantity2 & item2 ... &\\'
\ingredients[2] % [v1.1] number of ingredient columns 
{%
    \unit[200]{g} & flour\\
    \unit[4]{g} & baking soda\\
    \unit[150]{mL} & warm water\\
    some & olive oil\\
    some & salt\\
}


% ~~ PREPARATION ~~
% Steps as '\step{whatever the step is}'
% [until v1.1] Caution: for optimal floating, steps need to be at >= 2
% 				lines long! -> if one-liner, add a return ('\\') 
\preparation{%
    \step{Mix flour, baking soda and salt.}
    \step{Add the olive oil and enough water to make a soft batter that doesn't stick.}
    \step{Work the batter for about \unit[5]{mins}.}
    \step{Divide into portions and roll into breadsticks about \unit[20]{cm} long with a \unit[1]{cm} diameter.}
    \step{Lay it on a baking tray and let it rest for \unit[1]{hour}}
    \step{Bake for 20 mins at 200ºC (a bit less if your oven has a fan)}
} 


% ~~ SUGGESTIONS ~~
% Chef suggestions (optional)
\suggestion
[Add a little more flavour] % suggestion title
{ 
Before taking it all into the oven, add some herbs to make them even tastier. Oregano and black pepper are some of my favourites in this case!
}


\end{recipe}

% end of recipe file