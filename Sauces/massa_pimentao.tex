% Butternut Squash soup.tex
% recipe file for southern Portuguese fish soup

\begin{recipe}
[%
% ~~ RECIPE OVERVIEW: ~~
	preparationtime = {\unit[2]{days}},
	portion = {\unit[1]{jar}},
	source = {Ana},
	country = {Portugal},
	vegetarian = {yes},
	vegan = {yes}
]
%
% ~~ RECIPE NAME: ~~
{Massa de piment\~ao\\a.k.a. Red Pepper Sauce}


% ~~ PICTURES ~~
% Appear on top of the page (optional)
% Insert path\to\filename
\graph
{ 
    small=pimentos,     % small picture
    big=massa_pimentao  % big picture
}


% ~~ INTRODUCTION ~~
% History/anecdote on the recipe (optional)
\introduction
{ 
Made almost exclusively of red peppers, this is a classic sauce from Alentejo in Portugal. It can be used in pretty much anything, a teaspoon in any stew or stir-fry, or even as a spread on some toast. It is said that if you give it to someone from Alentejo, that hasn't been home in a long time, tears will soon run through their face. It's so delicious, it's no surprise!
}


% ~~ INGREDIENTS ~~
% List as '& quantity & item &\\'
% Caution: no return ('\\') after last item!
% [v1.1] Multi-columns: '& quantity1 & item1 && quantity2 & item2 ... &\\'
\ingredients[2] % [v1.1] number of ingredient columns 
{%
    4 & garlic cloves\\
    3 & red pepper\\
    \unit[100]{mL} & olive oil\\
    \unit[8]{tbsp} & salt\\
}


% ~~ PREPARATION ~~
% Steps as '\step{whatever the step is}'
% [until v1.1] Caution: for optimal floating, steps need to be at >= 2
% 				lines long! -> if one-liner, add a return ('\\') 
\preparation{%
    \step{Cut the red pepper in stripes of about \unit[3]{cm}.}
    \step{In a bowl, put the red peppers in a bowl in between layers of salt, with quite a bit of salt on top.}
    \step{Cover and leave for at least overnight but ideally for a couple of days.}
    \step{Rinse out the extra extra salt and the liquid from the peppers.}
    \step{Add the garlic and the olive oil and blend it all together until it's smooth.}
    \step{Put it in a jar and keep it refrigerated to use at any time!}
} 

\end{recipe}

% end of recipe file